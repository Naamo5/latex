\documentclass{standalone}

\usepackage{tikz}
\usetikzlibrary{scopes}
\usetikzlibrary{calc,patterns,decorations.pathmorphing,decorations.markings}

\begin{document}

\def\iangle{35} % Angle of the inclined plane

\def\down{-90}
\def\arcr{0.5cm} % Radius of the arc used to indicate angles

\begin{tikzpicture}[
    force/.style={>=latex,draw=blue,fill=blue},
    cine/.style={>=latex,draw=red,fill=red},
    axis/.style={densely dashed,gray,font=\small},
    M/.style={rectangle,draw,fill=lightgray,minimum size=0.5cm,thin},
    m/.style={circle,draw=black,fill=lightgray,minimum size=0.03cm,thin},
    plane/.style={draw=black,fill=blue!10},
    string/.style={draw=red, thick},
    pulley/.style={thick},
    spring/.style={thick,decorate,decoration={zigzag,pre length=0.3cm,post length=0.3cm,segment length=6}},
    damper/.style={thick,decoration={markings,
      mark connection node=dmp,
      mark=at position 0.5 with
      {
        \node (dmp) [thick,inner sep=0pt,transform shape,rotate=-90,minimum width=15pt,minimum height=3pt,draw=none] {};
        \draw [thick] ($(dmp.north east)+(2pt,0)$) -- (dmp.south east) -- (dmp.south west) -- ($(dmp.north west)+(2pt,0)$);
        \draw [thick] ($(dmp.north)+(0,-5pt)$) -- ($(dmp.north)+(0,5pt)$);
      }
    }, decorate},
]


    %%%
    % Free body diagram of m
    \coordinate (I) at (0.5, -0.5);
    %\node[m] (m) {};
    \coordinate (S) at (-2.5, 2.5);
    \coordinate (A) at (-1, 0);
    \coordinate (B) at (0, 1);
    \coordinate (C) at (-1, 2);
    \coordinate (D) at (-2, 1);
    \coordinate (E) at (-0.5, 0.5);
    \coordinate (F) at (-1.5, 1.5);


    \foreach \x in {(I), (S)}
    \draw[shift={\x}] node{$\bullet$};
    \draw (I)node[right]{I};
    \draw (S)node[above]{S};

    \draw [damper] (A) -- (D);
    \draw [spring] (B) -- (C);
    \draw (A) -- (B);
    \draw (C) -- (D);
    \draw (I) -- (E);
    \draw (F) -- (S);

    \path (S)++(-.7,.7)node{$\vec{\tau}(t)$};
    \draw[->] (-2.5,3) arc (90:220:.5cm);

    {[force,->]
        \draw (I.north) -- ++(0.7,1) node[right] {$\vec{F}(t)$};
        \draw (I.south) -- ++(0,-1) node[right] {$m\vec{g}$};
    }
    {[cine,->]
        \draw (S.east) -- ++(-1,-1) node[above left] {$\vec{v}(t)$};
    }

\end{tikzpicture}

\end{document}
