\documentclass[/home/francois/latex/report/main.tex]{subfiles}

\begin{document}

\newpage
\thispagestyle{plain}

~

\vfill

{ \setstretch{1.1}
	\subsection*{Author}
	François LE RALL – flr@kth.se\\
	School of Electrical Engineering and Computer Science\\
	KTH Royal Institute of Technology

	\subsection*{Host company}
	NoMagic Sp. z o. o.\\
	Rakowiecka 36 – 02-532 Warszawa – POLAND\\
	info@nomagic.ai

	\subsection*{Examiner}
	Joakim GUSTAFSON – KTH Royal Institute of Technology

	\subsection*{Supervisor}
	Piotr POLATOWSKI – NoMagic\\
	Patric JENSFELT – KTH Royal Institute of Technology
	~
}


\newpage
\thispagestyle{plain}
%%%%%%%%%%%%%%%%%%%%%%%%%%%%%%%%%%%%
%%  The English abstract          %%
%%%%%%%%%%%%%%%%%%%%%%%%%%%%%%%%%%%%
\chapter*{Abstract}
%%%%%%%%%%%%%%%%%%%%%%%%%%%%%%%%%%%%

The estimation of the mass of payloads, which are grasped by a vacuum cup gripper, may benefit several robotics applications, e.g.: object recognition, force-guided motions, hard failure detection. On the one hand, these applications require accurate, robust and fast estimation of the mass. On the other hand, measurement procedures might not break time-optimized motions of the robot. Existing methods, however, make use of \textit{exciting trajectories} that cannot be implemented in the context of industrial operations. Besides, no comparable projects have tackled the mechanical behavior of suction cup gripper.

This project proposes a method to measure the mass of the payload grasped by a robotic manipulator by estimating its inertial parameters. A mechanical model of the suction cup is derived to approximate the inertial parameters with a Recursive Total Least Squares algorithm. A calibration process, based on the dynamic model, is developed to initialize the inner properties of the tool and the vacuum cup.

The approach was evaluated on a database of a \textit{pick-and-place} robot operating in an \textit{e-commerce} warehouse. The results show that the system can estimate the mass with an average absolute error of $0.02 \ \si{\kilogram}$ and an average standard deviation of $0.015 \ \si{\kilogram\squared}$.

\subsection*{Keywords}

vacuum suction cup, robotic manipulator, inertial parameters, Least-Squares, rotational mass-spring-damper, dynamics, Recursive Total Least Squares

\newpage
\thispagestyle{plain}

\chapter*{Sammanfattning}

Uppskattningen av massan av nyttolaster, som greppas av en vakuumkoppgripare, kan gynna flera robotapplikationer, t.ex. objektigenkänning, kraftledda rörelser, detektering av hårt fel. Å ena sidan kräver dessa applikationer noggrann, robust och snabb massuppskattning. Å andra sidan kanske mätprocedurer inte bryter tidsoptimerade rörelser hos roboten. Befintliga metoder använder sig dock av \textit{spännande banor} som inte kan implementeras inom ramen för industriell verksamhet. Dessutom har inga jämförbara projekt tacklat sugkoppens mekaniska beteende.

Detta projekt föreslår en metod för att mäta massan på nyttolasten som grips av en robotmanipulator genom att uppskatta dess tröghetsparametrar. En mekanisk modell av sugkoppen är härledd för att ungefärliga tröghetsparametrarna med en Rekursiv Totala Minsta Kvadrat-algoritm. En kalibreringsprocess, baserad på den dynamiska modellen, utvecklas för att initialisera de inre egenskaperna hos verktyget och vakuumkoppen.

Metoden utvärderades på en databas över en \textit{pick-and-place} robot som fungerar i ett e-handelslager. Resultaten visar att systemet kan uppskatta massan med ett genomsnittligt absolutfel på $0{,}02 \ \si{\kilogram}$ och en genomsnittlig standardavvikelse på $0{,}015 \ \si{\kilogram\squared}$.

\subsection*{Nyckelord}

vakuum sugkopp, robotmanipulator, tröghetsparametrar, Minsta Kvadratmetoden, massa-fjäder-spjäll, dynamik, Rekursiv Totala Minsta Kvadratmetoden

\newpage
\thispagestyle{plain}
\chapter*{Acknowledgements}

First off, I want to thank the \textsc{Nomagic} team for a warm welcome during that mild Polish winter. Thank you Piotr for the advice, guidance, and everyday support. Thank Marek, Kacper, and Marek for the trust you have placed in me to produce my thesis study in the framework of an exceptional technological, scientific and multicultural project.

Thanks to Patric \textsc{Jensfelt}, for being my supervisor, for inculcating me with the \textit{spirit} of a Master Thesis, for the organization of the very valuable meetings and, for reading my misspelled reports.

Merci à mes parents pour le soutien qu'ils m'ont apporté tout au long de mes études, pour la patience dont ils ont fait preuve tout au long d'un parcours scolaire qui n'a fait que s'étirer dans le temps et dans l'espace. Je n'en serais pas là aujourd'hui si vous ne m'aviez pas enseigné qu'\textit{il n'y a qu'un seul moyen de réussir ; c'est de ne jamais se décourager}.

Merci à Mathieu pour l'aide qu'il m'a apportée tout au long de mes études à KTH. Tu m'as évité de tomber dans beaucoup de pièges et tu as grandement bonifié cette année à Stockholm.

A huge thanks to Hugo, Marc, Chloé, Laura, Jose, Ana, Riccardo, Panos, Agathe, Elena and others for being by my side during this year in Sweden. I \underline{do} believe that it was amazing.

\newpage

\chapter*{Acronyms}

\begin{acronym}[RDBMS]
\acro{CPU}{Central Processing Unit}
\end{acronym}


\newpage

\etocdepthtag.toc{mtchapter}
\etocsettagdepth{mtchapter}{subsection}
\etocsettagdepth{mtappendix}{none}
\thispagestyle{plain}
\tableofcontents

\newpage

\end{document}
