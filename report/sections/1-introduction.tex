\documentclass[/home/francois/latex/report/main.tex]{subfiles}

\begin{document}

\chapter{Introduction}
\label{chapter:introduction}

A growing number of logistics companies are incorporating robots to automate order fulfillment and warehousing processes. In storehouses with less and less human presence, automated systems run tasks as storing, moving, scanning and wrapping goods. Logistic robots guarantee a great uptime over manual labor while taking over alienating tasks. Warehousing companies garner the benefits of speed, efficiency and increased profits to remain competitive in a market driven by consumers wishing for faster and more reliable deliveries. The past decade has witnessed a very rapid increase in robots installed in 2017 with 69,000 units – a 162 \% increase over 2016 (26,294) according to the \ac{IFR} \cite{industrialRobot2018}. The same organization estimates that 485,300 unit sales will be sold for the period between 2019 and 2021.

The high expectations of this growing e-commerce market have raised challenges for autonomous systems. Traditional robots programmed with hard-coded moves quickly reach limits when it comes to dealing with order disparities in size, shape, weight, volume, and mechanical properties \cite{GQHuang2015}. Logistics robots might be able to adapt to a wide diversity of situations with fewer human-in-the-loop interactions. Those challenging tasks have required major research on various fields such as Object Recognition, Motion Control, Perception, Machine Learning, Reinforcement Learning.

When it comes to deliver goods or assemble a final product, moving an item from one place to another is a recurring task. The earliest known industrial robot is a \textit{programmable transfer machine} as known as \textsc{Unimation} patented by George \textsc{Devol} in 1954 \cite{Wallen2008}. Thus, the first robot aimed to perform the so-called \textit{pick-and-place}. This task consists of picking things up from one location, moving them to another location and placing them. A typical application comes with \ac{SMT} component placement systems. Robotic machines are used to place \ac{SMDs} onto a \ac{PCB}. They are used for high speed, high precision placing of a broad range of electronic components, like capacitors, resistors, integrated circuits onto the \ac{PCB}.

With the long-term goal of reducing costs and giving systems more flexibility, \textit{pick-and-place} systems tend to be more generic. Industrial robot systems aim not only to \textit{pick-and-place} at high velocity but also to be able to handle unknown objects. To execute such a global task, there are multiple challenges:
\begin{itemize}
    \item \textbf{picking}: the system has to localize precisely the object and find an efficient way to grasp it;
    \item \textbf{moving}: the robot try to move the object as fast as possible from source location to target location without losing it;
    \item \textbf{placing}: the manipulator should place the object at the desired position with the desired orientation;
    \item \textbf{overall}: the gripper should not damage the item by squeezing, scratching or shaking it.
\end{itemize}

To address those challenges, some ground-breaking research has been carried in the field of item recognition, object pose w.r.t.\ the gripper computation, item inertial parameters estimation, item measurement. To this end, researchers have focused on gathering information before, during and after the handling time as a source of improvement. Devices to capture data range from visual sensors (e.g., RGB cameras, depth cameras, laser line scanners, structured light sensors) to kinetic sensors (e.g., \ac{IMUs}, \ac{FT} sensors, scales) or more tool-specific sensors (e.g., pressure sensors, tactile sensors). Efforts towards object recognition mainly rely on visual information (with \textit{appearance-based} techniques \cite{Carmichael2002, Schmid1997, Viola2001, Murase1993} or \textit{purely geometric} approaches \cite{Hut1987, Sethi2001}). Features used to identify the object are usually related to the shape, the color or the texture of the object. A few projects by \textsc{Kubus} \textit{et al.} and \textsc{Farsoni} \textit{et al.} aim at estimating the inertial parameters of the object as features for recognition \cite{Kubus2008, Kubus2007, Kubus2014, Farsoni2018}. Inertial parameters comprise the mass, the position of the center of mass and the 6 independent inertia tensor coefficients. These methods utilize the kinematics measurement data and \ac{FT} measurement to estimate the inertial parameters. However, those approaches make use of specific excitation trajectories that barely fit with industry cycle time expectancies. They also do not take into account the mechanical link between the gripper and the item. On balance, it may be possible to estimate the mass only without impacting the time-optimized motion of the robotic arm.

Another challenge of robotics is to automate systems that can interact with humans. In that purpose, knowing accurately the dynamic parameters of a manipulated object is required for common coordination strategies in physical human-robot interaction. Recent projects (\textsc{Wilkening} \textit{et al.} \cite{Wilkening2016} and \textsc{Ćehajić} \textit{et al} \cite{Ćehajić2017}) try to conceive an identification strategy of object dynamics for physical human-robot interaction, which allows the tracking of desired human motion and inducing the motions necessary for parameter identification. To that end, they have to estimate inertial parameters of the payload on which unknown forces are applied.

This thesis will focus on the estimation of the mass of a manipulator payload in motion by means of a \ac{FT} sensor.

\section{Motivation}

Within the scope of \textit{pick-and-place}, the mass of the item grasped by the tool of the robotic manipulator is key information for several enhancements. Mass is a meaningful feature for object recognition. It is also a critical data for the dynamic of the system, hard failures detection and estimation of the pose of the item w.r.t.\ the end-effector.

Concerning item recognition, it appears that purely vision-based approaches may fail if the objects to be identified are similar w.r.t. their visual characteristics. For instance, there is indeed no visual difference between full and empty bottles of milk that might be treated differently by a robotic system for sorting recyclables. Recognition system based on inertial parameters (or combinations of visual information and inertial data) can tackle this bottleneck.
Visual recognition algorithms can be also computationally expensive. The use of multiple cameras and/or RGB-D cameras coupled with the generation of point clouds and/or \ac{CNN} requires complex hardware or time-consuming scanning. Time is the most expensive resource in the industry and unsatisfactory cycle times are the main reason why many bin-picking solutions have not found their way in warehouses or factory production lines.

Inertial parameters of the item grasped are highly valuable to perform highly-dynamic force-guided or force-guarded motions \cite{Garcia2006, KubusKroger2008}. In some instances, the mass of the object carried by the tool is not negligible compared to the end-effector and the dynamic is strongly affected.
Also, robotic manipulators are operating with unknown objects of different sizes, shapes, materials, textures and so on. As a consequence, they are more likely to yield hard failures: two items grasped instead of one, item lost during motion, item hit due to oversize. Inertial parameters estimation can induce the detection of such shortcomings.

Placing the object at the desired position with the desired orientation requires the knowledge of the pose of the item w.r.t.\ to the tool. Once the mass, the center of mass and the inertia tensor are computed, a rough measure of the size and/or shape, orientation and position of the item can be approximated.

\section{Challenges}

One can notice that whilst the topic of the thesis is to identify the mass of the manipulator payload, inertial parameters are at the very heart of the report. There are several rationales for this.

One initial approach consists in measuring the vertical force exerted by the \{tool + item\} set of bodies on the last joint of the robotic manipulator, inferring the mass of the ensemble and subtracting the mass of the tool. It appears that merely measuring the vertical force does not output conclusive results. First, the dynamics of the system while moving can be complex. The emergence of inertial side effects can make the measurement complicated. Secondly, \ac{FT} sensor can deliver a consistent measurement while static. However, without loss of generality, the performance can be reduced when the system is moving – due to the noise of measurement, or response time. Thirdly, the geometry of the system fluctuates because of the diversity of the grasping tools, the different natures of items handled or the different ways to grip the object. For instance, the center of mass of a narrow item picked by its extremity might not be directly below the sensor. Consequently, the mass of the payload is not directly linked to the vertical force.

Another approach would be to measure the force while the manipulator is static. Observations show that the system needs an overall of $0.5 \si{\second}$ to $1.0 \si{\second}$ to stabilize (cf. Section \ref{section:hardware} for more details on the setup). Nevertheless, as stated above, in the context of an industrial system, time is a crucial resource. A rule of thumb is that processes should not under any circumstances break the motion of the robot.

Thus, there is a need to estimate the mass of payloads while the robotic manipulator is moving. As mention before, mass, the center of mass and the inertia tensor of the object are required parameters to formulate the dynamic equations. Therefore, the project aims at estimating the mass of the item by approximating the whole inertial parameters.


\section{Formulation of the Problem}

This project aims to propose a method to estimate the mass of the manipulator payload in the context of \textit{pick-and-place} operations for industrial purposes. The solution might take into account constraints in time, in diversity of gripper tools and objects handled by the robotic arm.

Taking this objective into account, the research question that addresses this thesis is:

\begin{itemize}%[leftmargin=1em]
  \renewcommand{\labelitemi}{$\Rightarrow$}
 \item How to reliably measure the mass of a manipulator payload in motion by means of a \ac{FT} sensor?
\end{itemize}

In more detail, the main parts of the problem are:

\begin{itemize}
 \item Establish a mathematical framework to connect the kinematic and \ac{FT} measurements over some time to the inertial parameters of the payload. This section focuses on applying the \textsc{Newton-Euler} equations to a system comprised of the robotic manipulator, the gripper tool and the item grasped.
 \item Estimate intrinsic properties of the manipulator and tool system than cannot be measured at first blush. For example, inertia tensor and center of mass of the tool are unlikely to be provided by the constructor. What is more, the robot end-effector is very likely to be custom-developed.
 \item Produce a computational method to estimate the mass from the abovementioned equations. Several options have to be considered: design an online or an offline algorithm, take sample error into account.
\end{itemize}

\section{Scope}

Now that the topic of the thesis has been highlighted, the thesis scope can be divided into three main objectives:

\begin{itemize}
    \item First of all, the state of the art in inertial parameters estimation methods for robotic manipulators will be reviewed. Methods to estimate parameters from ordinary differential equations will be analyzed and one will be selected.
    \item Afterwards, rigid and \ac{RMSD} models of the \{tool + item\} system will be derived and methods to estimate additional parameters will be developed.
    \item Finally, the combination of the previously achieved approaches will be combined and evaluated over a set of data.
\end{itemize}

Equally important is to define the delimitations of the project:

\begin{itemize}
    \item The implementation of trajectories and motion planning is not part of the project. The solution adopted should not indeed depend on a specific motion of the robotic manipulator.
    \item The solution developed should be scalable on similar systems: robot manipulators with a \ac{FT} sensor mounted on the last joint. Items, gripper tool, \ac{FT} sensor, trajectories can be of different types.
    \item The project covers neither the improvement of the item recognition strategy nor the picking position prediction and grasping techniques.
\end{itemize}


\section{Report Outline}

The rest of the report is organized as follows: Chapter \ref{chapter:background} reviews the background pertaining to the thesis: dynamic models and \ac{RTLS} approach. Chapter \ref{chapter:method} presents the methods developed in this project. Chapter \ref{chapter:setup} depicts the experimental setup for testing the different methods. The experimental results are presented analyzed in chapter \ref{chapter:results}. Finally, in chapter \ref{chapter:conclusions}, conclusions are drawn and future work presentation closes the report.

\end{document}
