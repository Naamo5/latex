\documentclass[/home/francois/latex/report/main.tex]{subfiles}

\begin{document}

\chapter{Introduction}

A growing number of logistics companies are incorporating robots to automate order fulfillment and warehousing processes. In storehouses with less and less human presence, automated systems run tasks as storing, moving, scanning and wrapping goods. Logistic robots guarantee a great uptime over manual labor while taking over alienating tasks. Warehousing companies garner the benefits of speed, efficiency and increased profits to remain competitive in a market driven by consumers wishing for faster and more reliable deliveries. The past decade has witnessed a very rapid increase in robots installed in 2017 with 69,000 units – a 162 \% increase over 2016 (26,294) according to the \ac{IFR} \cite{industrialRobot2018}. The same organization estimates that 485,300 unit sales will be sold for the period between 2019 and 2021.

The high expectations of this growing e-commerce market has raised challenges for autonomous systems. Traditional robots programmed with hard-coded moves quickly reach limits when it comes to dealing with order disparities in size, shape, weight, volume and mechanical properties \cite{GQHuang2015}. Logistics robots might be able to adapt to a wide diversity of situations with fewer human-in-the-loop interaction. Those challenging tasks have required major research on various fields such as Object Recognition, Motion Control, Perception, Machine Learning, Reinforcement Learning.

\vspace{1cm}

\textit{TODO}

{\it
\begin{itemize}
	\item focus on pick \& place
	\begin{itemize}
		\item the first industrial robot was a pick and place (Programmed Transfer Article) robot \cite{Wallen2008}
		\item description of the pick and place task
		\item challenges (speed, unknown items, diversity of shapes, masses, materials)
	\end{itemize}
	\item item recognition
	\begin{itemize}
		\item most 3D rigid object recognition approaches rely on vision information
		\item limitation of visual approaches
	\end{itemize}
	\item identification of the mass by estimating the item inertial parameters. Explain here why it is fondamental to estimate the 10 inertial parameters to access the mass of the item.
\end{itemize}
}

This thesis will focus on the estimation of the mass of an item carried in motion by robotic manipulator by means of a \ac{FT} sensor.

\section{Motivation}

\textit{TODO}

{\it

\begin{itemize}
	\item inertial parameters estimation by means of a \ac{FT} sensor
	\begin{itemize}
		\item purely vision based approaches may fail if the objects to be identified are similar with respect to their visual characteristics
		\item combinations of visual and inertial features for recognition or classification methods
	\end{itemize}
	\item other purposes
	\begin{itemize}
		\item perform highly-dynamic force-guided or force-guarded motions \cite{Garcia2006, KubusKroger2008}
		\item detect hard failures (two items grasped, item lost during motion)
		\item estimate the position of the item
	\end{itemize}
\end{itemize}

%Companies running robotic manipulator in an industrial context might need to measure the mass of the item grasped for several purposes: adjust the kinematics of the robot, recognize the item, spot grasping error, gather information for the user or the customers. Most of the new generation robotic manipulators are equipped with a 6 \ac{DoF} \ac{FT} sensor but there is no mass estimation method built-in or pervasive in the robotic world.
}

\section{Challenges}

\textit{TODO}

{\it
To measure of the inertial parameters of an item while moving can be tough:

\begin{itemize}
	\item The dynamics of the system can be complex and the emergence of inertial side effect can make the measurement complicated.
	\item The diversity of the grasping tools, the different natures of item handled, the different ways to grip the object make the geometry of the system fickle.
	\item The \ac{FT} sensor can deliver a consistent measurement while static. However, the performance can be reduced when the system is moving (due to noise of measurement, response time, . . .).
	\item In the context of an industrial system, the computation time is a crucial point. A rule  thumb is that the running time should not break the motion of the robot.
\end{itemize}
}

% \ac{FT} sensor is mounted on the wrist of the robotic arm. It easily gets a weight measurement of the grasped item when the robot is static. A reliable static measure required the robotic manipulator to be perfectly standing and the F/T sensor to stabilize. This operation takes roughly 0.5s to be executed. It would be better to be able to measure the mass of an item while the robot is in motion. It may allow saving precious time in a grasping cycle and consequently increase productivity.

\section{Formulation of the Problem}

The research question that addresses this thesis is :

How to reliably measure the mass of a manipulator payload in motion by means of a \ac{FT} sensor?

\textit{TODO}

\section{Scope}

\textit{TODO}

{\it

% \begin{itemize}
% 	\item The implementation of trajectories and motion planning is not part of the project. The solution developed should be scalable on similar systems: robot manipulator with an F/T sensor mounted on the last joint. Items, grasping tool, F/T sensor, robotic arm trajectories can be of different types.
% 	\item The project does not include programming motion planning and control of the robotic arm. It does not cover the improvement of the item recognition strategy. Also, it does not comprise the creation of a dataset to evaluate the method.
% \end{itemize}
}


\section{Report Outline}

\textit{TODO}

{\it
The rest of the report is organized as follows: Chapter \ref{chapter:background} reviews the background pertaining to the thesis: dynamic model and \ac{RTLS} approach. Chapter \ref{chapter:method} presents the methods developped in this project: \ac{RMSD} system, signal filtering and \ac{RGTLS} with \ac{NCE}. Chapter \ref{chapter:setup} depicts the experimental setup for testing the different methods. The experimental results are presented analysed in the chapter \ref{chapter:results}. Finally, in the chapter \ref{chapter:conclusions}, conclusions are drawn and future work presentation closes the report.
}

\end{document}
