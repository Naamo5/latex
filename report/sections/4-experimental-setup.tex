\providecommand{\main}{..}

\documentclass[/home/francois/latex/report/main.tex]{subfiles}

\begin{document}

\chapter{Experimental Setup}
\label{chapter:setup}

This chapter covers additional details on the robotic manipulator setup, the dataset that will be used to test the proposed framework, as well as the requirements and assumptions used in the experiments.

\section{Dataset}

The performance of the estimation method is tested on a dataset of runs from a \textit{pick-and-place} system operating in a \textit{e-commerce} warehouse (cf. Figure \ref{fig:setup:maria}). The robot on site is a UR10 from \textsc{Universal Robot}\texttrademark \ and the hardware setup is depicted in the section \ref{section:hardware}.

\begin{figure}[h]
  \centering
  \includegraphics[scale=0.4]{\main/figures/maria.jpg}
  \caption{Robot operating in an \textit{e-commerce} warehouse. The item is grasped in the deep metal cart in the bottom right part of the picture. Then, the object is brought to an area where it is identified. After, it is dropped in a box for further delivery. On the picture, the robot is about to drop an item –apparently a DVD– in the cardboard delivery box. Image from \textsc{NoMagic}.}
  \label{fig:setup:maria}
\end{figure}

This robot packs thousands of items every week. The panel of products is rather larger with over $100{,}000$ different entries. A batch of 9 items highly picked is selected for the evaluation. They differ in terms of their mass and shape. Table \ref{tab:setup:items} depicts the set of items and some of their  properties.

\begin{table}[h]
  \begin{center}
    \renewcommand{\arraystretch}{1.2} % Default value: 1
    \begin{tabular}{l|c|c|c} % <-- Alignments: 1st column left, 2nd middle and 3rd right, with vertical lines in between
      \textbf{Designation} & \textbf{Mass ($\si{\kilogram}$)} & \textbf{Shape description} & \textbf{Number of runs}\\
      \hline
      \texttt{playstation\_plus}  & 0.0081 & Flat Cardboard & 306 \\
      \hline
      \texttt{sata\_480gb}  & 0.055 & Flat cuboid & 458\\
      \hline
      \texttt{canon\_pixma}  & 0.133 & Flat plastic ink cartridge & 256\\
      \hline
      \texttt{canon\_ink}  & 0.152 & Flat foiled ink cartridge & 141 \\
      \hline
      \texttt{samsung\_powerbank}  & 0.237 & Cardboard cuboid & 262\\
      \hline
      \texttt{samsung\_galaxy}  & 0.281 & Cardboard cuboid & 849 \\
      \hline
      \texttt{android\_mini\_pc}  & 0.393 & Cardboard cuboid & 535\\
      \hline
      \texttt{xiaomi\_9t}  & 0.508 & Cardboard cuboid & 265 \\
      \hline
      \texttt{apple\_tv}  & 0.818 & Cardboard cuboid & 110 \\
      \hline
    \end{tabular}
  \end{center}
  \caption{Selection of items for the evaluation of the framework. The number of runs represents the \textit{pick-and-place} cycles analysed per item. \label{tab:setup:items}}
\end{table}

\section{Calibration}

\subsection{Tool parameters}
\label{subsection:setup:calibration:tool}

The assessment of the calibration process is realized with the setup depicted in figure \ref{fig:background:stanislaw}. The estimation of the mass of the tool is compared with its mass measured with a scale. From the axial symmetries of the tool, it can be inferred that $c_1 = (0, 0, c_{1,z})$ in the tool frame. The value of the last coordinate can be measured by balancing the tool on a rod. The inertia tensor coefficients are harder to measure. However, considering the axes of symmetry, the shape of the inertia tensor can be expected. The order of magnitude of the inertia is computed with the formula of a thick-walled cylindrical tube with open ends, of inner radius $r_1$, outer radius $r_2$, length $h$ and mass $m$ (cf. equation \ref{eq:setup:tensor}). As every Fondamental Principle of Dynamics (FPD) equations are derived at the point A, the parallel axis theorem (cf. appendix \ref{appendix:notation:parallel_axis}) is used to compute $I_{A, tool}$.

\begin{equation}
  \label{eq:setup:tensor}
  I_{G_{tool}, tool} =
  \begin{pmatrix}
  \frac{1}{12} m(3(r_1^2+r_2^2) + h^2) & 0 & 0  \\
  0 & \frac{1}{12} m(3(r_1^2+r_2^2) + h^2) & 0 \\
  0 & 0 & \frac{1}{2} m(r_1^2+r_2^2)
  \end{pmatrix}
\end{equation}

Finally, the inertia tensor is estimated by measuring the length, the external and internal radii.

\begin{equation*}
  I_{A, tool} = I_{G_{tool}, tool} - m d^2 =
  \begin{pmatrix}
  5.8 \times 10^{-4} & 0 & 0  \\
  0 & 5.8 \times 10^{-4} & 0 \\
  0 & 0 & 0.001
  \end{pmatrix}
\end{equation*}

One can notice that the tool in figure \ref{fig:setup:maria} is not the same as the one of figure \ref{fig:background:stanislaw}. The production robot tool parameters are estimated with another system but not assessed by ground-truth measurement. The suction cup is precisely the same as the one used on the production site so the calibration can be used for further evaluation of the method.

\subsection{Suction cup properties}

To estimate the spring constant and the damper coefficient, the standard item is preferably a heavy and cuboid item ($\approx 0.5-1 \si{\kilogram}$). An iPad box filled with expansive foam (cf. figure \ref{fig:background:stanislaw}) is used for that purpose. The shape of box is $0.055 \times 0.140 \times 0.210 \si{\meter}$ and its mass ($0.519 \si{\kilogram}$) is well distributed.

\end{document}
