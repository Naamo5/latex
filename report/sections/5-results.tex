\providecommand{\main}{..}

\documentclass[/home/francois/latex/report/main.tex]{subfiles}

\begin{document}

\chapter{Experimental results}
\label{chapter:results}

This chapter presents the evaluation of the approaches developed. First, the outcomes of the signal filters are depicted, secondly, the performance of the calibration process is outlined, finally, results of the rigid and \ac{RMSD} models are compared as well as the simple \ac{LS} and \ac{RTLS} ones.

\section{Signals pre-processing}
\label{section:results:pre-processing}

The performance of the filters is compared with non-filtered signals (\ac{FT} measurements and orientation of the \ac{TCP}) and with a basic incremental derivation.

\subsection{\textsc{Euler}'s angles continuity filter}

The orientation of the \ac{TCP} output by the robot is converted to quaternion and then back to \textsc{Euler} angles (cf. Subsection \ref{subsection:method:euler-filter}). The output of the quaternion filter is depicted in the figure \ref{fig:results:quat-conv}. For instance, at $t = 3.0 \ \si{\second}$, the raw orientation is $\Theta_{raw} = (2.0, -2.3, -0.15)$ and the filter converts to  $\Theta_{quat} = (3.0, 0.0, 1.5)$ which is more readable.

As it can be noticed in the figure \ref{fig:results:quat-conv}, the orientation is not continuous and jump form $-\pi$ to $\pi$. Therefore, the continuity algoritm (cf. Algorithm \ref{alg:method:continuity}) is applied and the filtered signal is shown in the figure \ref{fig:results:orientation-filtered}.

\begin{figure}[h]
  \centering
  \includegraphics[scale=0.18]{\main/figures/maria_orientation_quat_conv.pdf}
  \caption{The orientation is converted to quaternion, then to a rotation matrix and back to \textsc{Euler} angle to ensure the unicity of the triplet.}
  \label{fig:results:quat-conv}
\end{figure}

\begin{figure}[H]
  \centering
  \includegraphics[scale=0.18]{\main/figures/maria_orientation_filtered.pdf}
  \caption{The continuity of the orientation signal is ensured by the filtering with the Algorithm \ref{alg:method:continuity}.}
  \label{fig:results:orientation-filtered}
\end{figure}

\subsection{\textsc{Savitzky–Golay} filter}

The output of the \textsc{Savitzky–Golay} filter for the derivation of the linear speed and angular acceleration are presented hereafter. The comparison is made with a classical iterative derivation ($f'(t) \approx  [f(t+\delta) - f(t)] / \delta$). The filter avoids the huge spikes generated by the measurement noise. Ultimately, it behaves analogously to a low-pass filter.


\begin{figure}[H]
  \centering
  \includegraphics[scale=0.17]{\main/figures/maria_linear_speed.pdf}
  \caption{First derivative of the position of the \ac{TCP} derived with a \textsc{Savitzky–Golay} filter. The filter works with a window of size 19 ($\approx 0.13 \ \si{\second}$) and a polynomial of degree 3.}
  \label{fig:results:linear-speed-filtered}
\end{figure}

\begin{figure}[H]
  \centering
  \includegraphics[scale=0.17]{\main/figures/maria_angular_acceleration.pdf}
  \caption{Second derivative of the orientation of the \ac{TCP} derived with a \textsc{Savitzky–Golay} filter. The filter works with a window of size 45 ($\approx 0.3 \ \si{\second}$) and a polynomial of degree 4.}
  \label{fig:results:angular-acceleration-filtered}
\end{figure}

\subsection{Low-pass filter}

The \ac{FT} signals is filtered with a \textsc{Butterworth} low-pass filter. The noise has two characteristics: its frequency is around $50 \ \si{\hertz}$ and its amplitude increases consequently when the signal varies. The output of the filter for \ac{FT} signals is depicted in figure \ref{fig:results:force-torque}. The filtered signals show less noise and thus meet the expectations.

\begin{figure}[h]
\centering
\begin{subfigure}{0.6\textwidth}
\centering
\includegraphics[scale=0.18]{\main/figures/maria_force.pdf}
\caption{Force acting on the tool.}
\label{fig:results:force}
\end{subfigure}
\begin{subfigure}{0.6\textwidth}
\centering
\includegraphics[scale=0.18]{\main/figures/maria_torque.pdf}
\caption{Torque acting on the tool.}
\label{fig:results:torque}
\end{subfigure}
\caption{Torque acting on the tool measured by a Gamma \ac{FT} sensor. The  filter applied is a \textsc{Butterworth} low-pass filter of degree 8 with a cut of frequency of $18 \ \si{\hertz}$. The filter is applied two times –forward and backward– to have a zero phase output. Consequently the cut of frequency is $36 \ \si{\hertz}$. \label{fig:results:force-torque}}
\end{figure}

\section{Calibration of the gripper}
\label{section:results:calibration}

\subsection{Inertial parameters of the tool}

As mentioned in the chapter \ref{chapter:method}, the inertial parameters of the tool are estimated using the rigid model defined in section \ref{section:background:rigid}. The calibration is tested on the robot depicted in figure \ref{fig:background:stanislaw}. From the equation of torque, the parameters are optimized such that the estimated torque signals fit the measured ones. The output of the calibration is presented in table \ref{tab:results:calibration-tool}.

\begin{table}[h]
  \begin{center}
    \renewcommand{\arraystretch}{1.2} % Default value: 1
    \begin{tabular}{l|c|c} % <-- Alignments: 1st column left, 2nd middle and 3rd right, with vertical lines in between
      \textbf{Parameters} & \textbf{Estimated value} & \textbf{Measured value}\\
      \hline
      Mass ($\si{\kilogram}$)  & 0.492 & 0.505 \\
      \hline
      Center of mass $c_x$ ($\si{\meter}$)  & 0.001 & 0.0 \\
      \hline
      Center of mass $c_y$ ($\si{\meter}$)  & 0.0003 & 0.0 \\
      \hline
      Center of mass $c_y$ ($\si{\meter}$)  & 0.181 & 0.19 \\
      \hline
      Inertia $I_{xx}$ ($\si{\kilogram\meter\squared}$)  & $8.7 \times 10^{-4}$ & $5.8 \times 10^{-4}$ \\
      \hline
      Inertia $I_{xy}$ ($\si{\kilogram\meter\squared}$)  & $2.3 \times 10^{-5}$ & 0.0 \\
      \hline
      Inertia $I_{xz}$ ($\si{\kilogram\meter\squared}$)  & $1.8 \times 10^{-5}$ & 0.0 \\
      \hline
      Inertia $I_{yy}$ ($\si{\kilogram\meter\squared}$)  & $5.3 \times 10^{-4}$ & $5.8 \times 10^{-4}$ \\
      \hline
      Inertia $I_{yz}$ ($\si{\kilogram\meter\squared}$)  & $1.1 \times 10^{-5}$ & 0.0 \\
      \hline
      Inertia $I_{zz}$ ($\si{\kilogram\meter\squared}$)  & 0.0017 & 0.001 \\
      \hline
    \end{tabular}
  \end{center}
  \caption{Comparison between the estimated and measured inertial parameters of the tool.\label{tab:results:calibration-tool}}
\end{table}

The calibration of the mass is not as precise as the accuracy of the \ac{FT} sensor (cf. Table \ref{tab:background:ft-sensor}). It would be better to estimate the mass separately. It is possible to measure the force with the tool upside down (the \ac{FT} sensor at the bottom) and compute the difference with the standing position. Then the mass line in the matrix equation of the rigid model (cf. equation \ref{eq:appendix:rigid}) could be merely removed. The estimation of the center of mass and the inertia are satisfactory considering that the ground-truth values are hard to measure.

To confirm the soundness of the calibration, the estimated torque and the measured torque are compared on the figure \ref{fig:results:calibration:torque}.

\begin{figure}[h]
\centering
\begin{subfigure}{0.49\textwidth}
\centering
\includegraphics[scale=0.25]{\main/figures/load_free_torque_x_comparison.pdf}
\caption{Torque over $\overrightarrow{x_0}$.}
\label{fig:results:calibration:torque-x}
\end{subfigure}
\begin{subfigure}{0.49\textwidth}
\centering
\includegraphics[scale=0.25]{\main/figures/load_free_torque_y_comparison.pdf}
\caption{Torque over $\overrightarrow{y_0}$.}
\label{fig:results:calibration:torque-y}
\end{subfigure}
\begin{subfigure}{0.51\textwidth}
\centering
\includegraphics[scale=0.25]{\main/figures/load_free_torque_z_comparison.pdf}
\caption{Torque over $\overrightarrow{z_0}$.}
\label{fig:results:calibration:torque-z}
\end{subfigure}
\caption{Comparison of the estimated and measured torque for the calibration of the inertial tool parameters. In that case, the robot manipulator goes in different directions, at different speeds, rotates such that it excites every axes of inertia of the tool. \label{fig:results:calibration:torque}}
\end{figure}

Also, the figure \ref{fig:results:calibration:angle} shows the orientation of the item about the main frame. The frame of the item and the tool are meant to be aligned when picking. That is why the initial angle is $\pi$.

\begin{figure}[h]
  \centering
  \includegraphics[scale=0.3]{\main/figures/angle_y.pdf}
  \caption{Orientation of the item in the base\_link frame about the $y$ axis. From comparison with observation, the order of magnitude seems correct.}
  \label{fig:results:calibration:angle}
\end{figure}

\subsection{Spring constant and damper coefficient}

Once the tool parameters have been calibrated, the \ac{RMSD} model comes into play to estimate the vacuum cup properties. The equation of the torque from \ac{FPD} applied to the tool \{1\} and the item \{2\} are combined to optimize the two parameters. The vacuum gripper grasped the standard item described in section \ref{subsection:setup:calibration:tool} right in the middle of a face. The inertial properties of the item are entered in the model, as well as the distance between the \ac{TCP} and the center of mass of the item. The larger dimension of the iPad box is lined up the $\overrightarrow{x_0}$. First, the robot arm moves horizontally along the $y$-axis. From that kinetic data, $k$ and $\lambda$ are estimated (cf. table \ref{tab:results:calibration-cup}). The comparison of the modeled and measured torque is presented in figure \ref{fig:results:calibration:suction-torque-x}.

\begin{table}[h]
  \begin{center}
    \renewcommand{\arraystretch}{1.2} % Default value: 1
    \begin{tabular}{l|c} % <-- Alignments: 1st column left, 2nd middle and 3rd right, with vertical lines in between
      \textbf{Parameters} & \textbf{Estimated value} \\
      \hline
      Spring constant $k$ ($\si{\newton\meter\per\radian}$) & 0.016\\
      \hline
      Damper coefficient $\lambda$ ($\si{\newton\meter\second\per\radian}$) & 0.041 \\
      \hline
    \end{tabular}
  \end{center}
  \caption{Comparison between the estimated and measured inertial parameters of the tool.\label{tab:results:calibration-cup}}
\end{table}

\begin{figure}[h]
  \centering
  \includegraphics[scale=0.35]{\main/figures/calib_torque_x.pdf}
  \caption{Comparison of the estimated and measured torque over the $x$-axis for the calibration of the suction cup properties.}
  \label{fig:results:calibration:suction-torque-x}
\end{figure}

With no theoretical values to compare the results, the robot executes a similar motion about the $x$-axis with the same configuration. The larger dimension of the iPad box is still lined up the $\overrightarrow{x_0}$. This explains why the amplitude of oscillation is smaller. The estimated torque is compared with the measured one (cf. Figure \ref{fig:results:calibration:suction-torque-y})in order to validate the abovementioned results.

\begin{figure}[h]
  \centering
  \includegraphics[scale=0.35]{\main/figures/calib_torque_y.pdf}
  \caption{Validation of the estimated and measured torque over the $y$-axis.}
  \label{fig:results:calibration:suction-torque-y}
\end{figure}

The estimated torque does not perfectly fit the curve in figure \ref{fig:results:calibration:suction-torque-x} and \ref{fig:results:calibration:suction-torque-y}. There is probably a tradeoff in the optimization between the period of oscillation and amplitude. The \ac{LS} methods are likely to favor a model that fits the frequency of the output rather than the amplitude. Indeed, if the amplitude is perfectly fitted but the period is slightly different, the least-squares differences are significant when the estimated signal shifts with time due to the unsynchronization.

\section{Least-Squares method and rigid model}

\subsection{Cart to box setup}

In a first attempt at estimating the mass of the item, the framework is tested with the item set listed in table \ref{tab:setup:items}. For each run, the mass is estimated between the moment the item is grasped to the moment it goes out of the volume above the source cart with the rigid model. Two methods are compared in this evaluation: the \textsc{Scipy} \texttt{scipy.optimize.least\_squares} and the \ac{SVD} algorithm (cf. \ref{subsection:ls}). Since the mass of an \ac{SKU} can differ due to product modification or discount, the ground-truth mass is estimated while the item is standing still during the measurement process. In this way, the estimated mass can be compared to the measured one.

\begin{figure}[H]
  \centering
  \includegraphics[scale=0.3, angle=-90]{\main/figures/overall.png}
  \caption{Histograms and estimated normal distributions of the estimated mass of the item set via \textsc{Scipy} and \ac{SVD} method. The black bell curve is the normal distribution of the standstill measurement of the ground-truth value. The \ac{SVD} shows better result for lighter items.}
  \label{fig:results:rigid-ls}
\end{figure}

As it can be seen from the figure \ref{fig:results:rigid-ls}, the average absolute error with the \textsc{Scipy} method for the whole item set is $0.0154 \ \si{\kilogram}$ and $0.0104 \ \si{\kilogram}$ for the \ac{SVD}. The average standard deviation is $0.0143 \ \si{\kilogram\squared}$ for the first and $0.0149 \ \si{\kilogram\squared}$ for the second. In this specific case, the \ac{SVD} algorithm performs better for a similar computation time.

One can notice, the estimated normal distribution of the standstill measurements and the thesis procedure ones are very close together. Therefore, the performance of the system is almost at the same level as the standstill measurement. Also, in the specific case of single-\ac{SKU} picking, the precision of the outputs is high enough to detect if the robot picks two items. As a matter of fact, over $\approx 0.100 \ \si{\kilogram}$, the process can detect such hard failures, as the bell curve of a double-pick item does not overlap the curve bell of a single-pick.

At this point in the project, the performance has been considered high enough for this particular case of \textit{pick-and-place}. The following evaluations are therefore done on the robot of the figure \ref{fig:background:stanislaw}. As explained in the abovementioned section, the robot pick from a tote and place in a tote. The trajectories are shorter, the robot is faster and does not offer a large time-window to record the data.

\subsection{Tote to tote setup}
\label{results:tote}


For this test, a box of $0.480 \ \si{\kilogram}$ has been picked and placed. The estimation setup is the same as for the previous test. Results are depicted in the figure \ref{fig:results:rigid-ls-tote}. The \ac{SVD} and classic \ac{LS} method appears to perform almost the same. The average absolute error is above $0.200 \ \si{\kilogram}$ for both and the standard deviation $0.090 \ \si{\kilogram\squared}$ for the first and $0.076 \ \si{\kilogram\squared}$ for the second. The significant difference in performance is due to that the speed is higher and the recording duration shorter.

\begin{figure}[h]
  \centering
  \includegraphics[scale=0.18, angle=0]{\main/figures/totes_ipad.png}
  \caption{Histograms and estimated normal distributions of the estimated mass of $0.480 \ \si{\kilogram}$ boxes with the tote to tote usecase (112 runs). In blue, the \textsc{Scipy} \ac{LS} function, in orange, the \ac{SVD} method.}
  \label{fig:results:rigid-ls-tote}
\end{figure}

An evaluation with the \ac{RMSD} model and the \ac{RTLS} algorithm is conducted with half a thousand picks of the same box. As it can be seen on the figure \ref{fig:results:rmsd-rtls-tote}, the average absolute error is lower with $0.063 \si{\kilogram}$ and the output is more grouped as the standard deviation is $0.040 \ \si{\kilogram\squared}$.

\begin{figure}[h]
  \centering
  \includegraphics[scale=0.18, angle=0]{\main/figures/totes_ipad_rtls.png}
  \caption{Histograms and estimated normal distributions of the estimated mass of $0.480 \ \si{\kilogram}$ boxes with the tote to tote usecase (457 runs). The mass is estimated with a \ac{RMSD} model fitted with \ac{RTLS}}
  \label{fig:results:rmsd-rtls-tote}
\end{figure}

Considering the bell curve, the precision of the whole process should be sufficient to detect the \textit{two-items-grasped} hard failure introduced above. The performance is insufficient to use the mass as a feature for item detection. It might be used as a validation step but could not be part of a \ac{CNN} input.

Thus, there is still plenty of scope for further development. Several weak points of the method have been spotted. First, the data recording may sometimes start when the item is still touching another one in the tote or oscillating. In the model, the initial condition for the orientation of the item is the same as the tool with no speed. The method can be improved by detecting the best moment to start caching the signals. Secondly, the computation of the orientation of the item is done with a \textsc{Matlab} script using \textsc{Simulink} and called with a \textsc{Python2Matlab} engine. This is not cost-effective and time-efficient at all. Thirdly, the framework does not output any indications about the confidence interval. A solution would be to use a \ac{RGTLS} with \ac{NCE}. On the one hand, it could improve the performance of the curve fitting step by better estimating the error. On the second hand, it could output a reliability factor base on the \ac{NCE}.

\end{document}
