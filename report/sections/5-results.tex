\providecommand{\main}{..}

\documentclass[/home/francois/latex/report/main.tex]{subfiles}

\begin{document}

\chapter{Experimental results}
\label{chapter:results}

This chapter presents the evaluation of the approaches developed. First, the outcome of the signal filters are depicted, secondly the performance of the calibration process are outlined, finally results of the rigid and \ac{RMSD} models are compared as well as the simple \ac{LS} and \ac{RTLS} ones.

\section{Signals pre-processing}
\label{section:results:pre-processing}

The performance of the filters are compared with a non-filtered signals (\ac{FT} measurements and orientation of the \ac{TCP}) and with a basic incremental derivation.

\subsection{\textsc{Euler}'s angles continuity filter}

The orientation of the \ac{TCP} output by the robot is converted to quaternion and then back to \textsc{Euler} angles (cf. Subsection \ref{subsection:method:euler-filter}). The output of the quaternion filter is depicted in the figure \ref{fig:results:quat-conv}. For instance, at $t = \si{\second}$, the raw orientation is $\Theta_{raw} = (2.0, -2.3, -0.15)$ and the filter converts to  $\Theta_{quat} = (3.0, 0.0, 1.5)$ which is more readable.

As it can be noticed in the figure \ref{fig:results:quat-conv}, the orientation is not continuous and jump form $-\pi$ to $\pi$. Therefore, the continuity algoritm (cf. Algorithm \ref{alg:method:continuity}) is applied and the filtered signal is shown in the figure \ref{fig:results:orientation-filtered}.

\begin{figure}[h]
  \centering
  \includegraphics[scale=0.22]{\main/figures/maria_orientation_quat_conv.pdf}
  \caption{The orientation is converted to quaternion, then to a rotation matrix and back to \textsc{Euler} angle to ensure the unicity of the triplet.}
  \label{fig:results:quat-conv}
\end{figure}

\begin{figure}[h]
  \centering
  \includegraphics[scale=0.22]{\main/figures/maria_orientation_filtered.pdf}
  \caption{The continuity of the orientation signal is ensured by the filtering with the Algorithm \ref{alg:method:continuity}.}
  \label{fig:results:orientation-filtered}
\end{figure}

\subsection{\textsc{Savitzky–Golay} filter}

The output of the \textsc{Savitzky–Golay} filter for the derivation of the linear speed and angular acceleration are presented here after. The comparison is made with a classical iterative derivation ($f'(t) \approx  [f(t+\delta) - f(t)] / \delta$). The filter avoid the huge spikes generated by the measurement noise. Ultimately, it behaves analogously to a low-pass filter.


\begin{figure}[h]
  \centering
  \includegraphics[scale=0.22]{\main/figures/maria_linear_speed.pdf}
  \caption{First derivative of the position of the \ac{TCP} derived with a \textsc{Savitzky–Golay} filter. The filter works with a window of size 19 ($\approx 0.13 \si{\second}$) and a polynomial of degree 3.}
  \label{fig:results:linear-speed-filtered}
\end{figure}

\begin{figure}[h]
  \centering
  \includegraphics[scale=0.22]{\main/figures/maria_angular_acceleration.pdf}
  \caption{Second derivative of the orientation of the \ac{TCP} derived with a \textsc{Savitzky–Golay} filter. The filter works with a window of size 45 ($\approx 0.3 \si{\second}$) and a polynomial of degree 4.}
  \label{fig:results:angular-acceleration-filtered}
\end{figure}

\subsection{Low-pass filter}

The \ac{FT} signals is filtered with a \textsc{Butterworth} low-pass filter. The noise has two characteristics: its frequency is around $50 \si{\hertz}$ and its amplitude increases consequenty when the signal varies. The output of the filter for \ac{FT} signals is depicted on the figure \ref{fig:results:force-torque}. The filtered signals show less noise and thus meet the expectations.

\begin{figure}[h]
\centering
\begin{subfigure}{0.6\textwidth}
\centering
\includegraphics[scale=0.22]{\main/figures/maria_force.pdf}
\caption{Force acting on the tool.}
\label{fig:results:force}
\end{subfigure}
\begin{subfigure}{0.6\textwidth}
\centering
\includegraphics[scale=0.22]{\main/figures/maria_torque.pdf}
\caption{Torque acting on the tool.}
\label{fig:results:torque}
\end{subfigure}
\caption{Torque acting on the tool measured by a Gamma \ac{FT} sensor. The  filter applied is a \textsc{Butterworth} low-pass filter of degree 8 with a cut of frequency of $18 \si{\hertz}$. The filter is applied two times –forward and backward– to have a zero phase output. Consequently the cut of frequency is $36 \si{\hertz}$. \label{fig:results:force-torque}}
\end{figure}

\section{Calibration of the gripper}
\label{section:results:calibration}

\subsection{Inertial parameters of the tool}

As mentioned in the chapter \ref{chapter:method}, the inertial parameters of the tool are estimated using the rigid model defined in section \ref{section:background:rigid}. The calibration is tested on the robot depicted on figure \ref{fig:background:stanislaw}. From the equation of torque, the parameters are optimized such that the estimated torque signals fit the measured ones. The output of the calibration is presented in the table \ref{tab:results:calibration-tool}.

\begin{table}[h]
  \begin{center}
    \renewcommand{\arraystretch}{1.2} % Default value: 1
    \begin{tabular}{l|c|c} % <-- Alignments: 1st column left, 2nd middle and 3rd right, with vertical lines in between
      \textbf{Parameters} & \textbf{Estimated value} & \textbf{Measured value}\\
      \hline
      Mass ($\si{\kilogram}$)  & 0.492 & 0.505 \\
      \hline
      Center of mass $c_x$ ($\si{\meter}$)  & 0.001 & 0.0 \\
      \hline
      Center of mass $c_y$ ($\si{\meter}$)  & 0.0003 & 0.0 \\
      \hline
      Center of mass $c_y$ ($\si{\meter}$)  & 0.181 & 0.19 \\
      \hline
      Inertia $I_{xx}$ ($\si{\kilogram\meter\squared}$)  & $8.7 \times 10^{-4}$ & $5.8 \times 10^{-4}$ \\
      \hline
      Inertia $I_{xy}$ ($\si{\kilogram\meter\squared}$)  & $2.3 \times 10^{-5}$ & 0.0 \\
      \hline
      Inertia $I_{xz}$ ($\si{\kilogram\meter\squared}$)  & $1.8 \times 10^{-5}$ & 0.0 \\
      \hline
      Inertia $I_{yy}$ ($\si{\kilogram\meter\squared}$)  & $5.3 \times 10^{-4}$ & $5.8 \times 10^{-4}$ \\
      \hline
      Inertia $I_{yz}$ ($\si{\kilogram\meter\squared}$)  & $1.1 \times 10^{-5}$ & 0.0 \\
      \hline
      Inertia $I_{zz}$ ($\si{\kilogram\meter\squared}$)  & 0.0017 & 0.001 \\
      \hline
    \end{tabular}
  \end{center}
  \caption{Comparison between the estimated and measured inertial parameters of the tool.\label{tab:results:calibration-tool}}
\end{table}

The calibration of the mass is not as precise as the accuracy of the \ac{FT} sensor (cf. Table \ref{tab:background:ft-sensor}). In fact, it would be better to estimate the mass separately. It is possible to measure the force with the tool upside down (the \ac{FT} sensor at the bottom) and compute the difference with the standing position. Then the mass line in the matrix equation of the rigid model (cf. equation \ref{eq:appendix:rigid}) could be merely removed. The estimation of the center of mass and the inertia are satisfying considering that the ground-truth values are hard to measure.

In order to confirm the soundness of the calibration, the estimated torque and the measured torque are compared on the figure \ref{fig:results:calibration:torque}.

\begin{figure}[h]
\centering
\begin{subfigure}{0.49\textwidth}
\centering
\includegraphics[scale=0.25]{\main/figures/load_free_torque_x_comparison.pdf}
\caption{Torque over $\overrightarrow{x_0}$.}
\label{fig:results:calibration:torque-x}
\end{subfigure}
\begin{subfigure}{0.49\textwidth}
\centering
\includegraphics[scale=0.25]{\main/figures/load_free_torque_y_comparison.pdf}
\caption{Torque over $\overrightarrow{y_0}$.}
\label{fig:results:calibration:torque-y}
\end{subfigure}
\begin{subfigure}{0.51\textwidth}
\centering
\includegraphics[scale=0.25]{\main/figures/load_free_torque_z_comparison.pdf}
\caption{Torque over $\overrightarrow{z_0}$.}
\label{fig:results:calibration:torque-z}
\end{subfigure}
\caption{Comparison of the estimated and measured torque for the calibration of the inertial tool parameters. In that case, the robot manipulator goes in different directions, at different speeds, rotates such that it excites every axes of inertia of the tool. \label{fig:results:calibration:torque}}
\end{figure}

\subsection{Spring constant and damper coefficient}

\begin{figure}[h]
  \centering
  \includegraphics[scale=0.35]{\main/figures/calib_torque_x.pdf}
  \caption{Comparison of the estimated and measured torque over the $x$-axis for the calibration of the suction cup properties.}
  \label{fig:results:calibration:suction-torque-x}
\end{figure}

\begin{figure}[h]
  \centering
  \includegraphics[scale=0.35]{\main/figures/calib_torque_y.pdf}
  \caption{Validation of the estimated and measured torque over the $y$-axis.}
  \label{fig:results:calibration:suction-torque-y}
\end{figure}


\section{One Body Model}

\textit{TODO}

\section{RMSD Model}

\textit{TODO}

\end{document}
